\documentclass[]{article}
\usepackage{graphicx}
\usepackage{apacite}
\usepackage{a4wide}
%opening
\title{Stock Analysis: Microsoft (MSFT)}
\author{Kuttner Maximilian, Sünderhauf Lukas, Köfer Robin}

\begin{document}

\maketitle
\bibliographystyle{apacite}


\begin{abstract}
\noindent In the following we are going to present, why \textbf{Microsoft Corporations (MSFT)} constitutes a \textbf{buying}-opportunity as of September 5, 2020.
\end{abstract}

\section{Company Overview}
Microsoft Corp. engages in the development and support of computer software, services, devices, and solutions. It operates through the following business segments: \textbf{Productivity and Business Processes, Intelligent Cloud} and \textbf{More Personal Computing}.\\\\
The Productivity and Business Processes segment consists of products and services in Microsoft's portfolio of productivity, communication, and information services, spanning a variety of devices and platforms like MS Office (comercial and consumer), Exchange, SharePoint, Microsoft Teams, but also LinkedIn and cloud-based enterprise ressource planning solutions (ERP) for businesses.\\\\
The Intelligent Cloud segment consists of their public, private, and hybrid server products and cloud services that can power modern business and developers including MS Azure, SQL Server, Visual Studio, System Center, Github or various support services for Microsoft Consulting Services.\\\\
The More Personal Computing segment encompasses products and services geared towards the interests of end users, developers, and IT professionals across all devices. This segment comprises products like Windows, devices like Surface and other pc accessories and gaming with various XBox-hardware, -content and -services.\\\\
The company was founded by Paul Gardner Allen and William Henry Gates III in 1975, is headquartered in Redmond, WA. and went public in 1986. Its current CEO is Satya Nadella.\cite{wiki:xxx}
\clearpage
\section{Key Financial Figures}
The company is listed on the NASDAQ stock exchange. Key stock figures are:
\begin{itemize}
	\item Market Capitalisation(B): \$ 1.672
	\item Enterprise Value (B): \$ 1.599
	\item Sales(B): \$ 143.0
	\item Dividend (Ann): \$ 2.04
	\item Div Yld: 0.9\%
\end{itemize}
Microsoft is currently the second largest company by market capitalization behind Amazon.
\section{Price Data}
\begin{figure}[h]
	\centering
	\includegraphics[scale=0.35]{"MSFT_YahooFinanceChart.png"}
	\caption{MSFT stock price}
	\cite{yahoo!finance_2020}
	\label{fig:meine-grafik}
\end{figure}
\noindent As seen in the figure above, the stock price of MSFT has been steadily rising during the past years. The stock experienced a rapid drop in price, caused by the COVID-19 pandemic from which it quickly recovered. Some key price figures:
\begin{itemize}
	\item Current Price: \$217.30
	\item 52 Wk Range: \$132.52 - 232.86
	\item Avg. Daily Vol (3m): 34.32(M)
	\item Short Interest: 0.4\%
\end{itemize}
\noindent Note that the short interest, i.e. the percentage of orders, which are short MSFT, is not even at 1\%. This is an indicator for a strong bullish sentiment. This might be a good opportunity for momentum/trend-following investors.
\section{Earnings}
\noindent Since price-action alone is not sufficient as an indicator for investors, we need to consider profitability measures aswell in order to determine if the stock is worth buying. To evaluate the profitability of MSFT we need to consider various profitability factors and metrics. First, we look at the earnings per share (EPS). EPS indicates how much money a company makes for each share of its stock and is a widely used metric for corporate profits. A higher EPS indicates a higher value, since investors are willing to pay more for a company with higher profits.\cite{chen_2020}
Estimates for future EPS and future price estimates were aggregated for multiple analysts [ressource: FactSet]:
\begin{itemize}
\item EPS (FY0): \$ 5.76
\item EPS (FY1): \$ 6.45
\item EPS (FY2): \$ 7.33
\item coverage: 31 Analysts
\item Target Price: \$233.13
\end{itemize}
One can observe that the estimates for Microsoft's future EPS are predicted to rise in the future, which inturn means that investors see MSFT's earnings rising and are willing to pay more for MSFT stock in the future. Moreover, the average EPS of the NASDAQ100 index is \$1.29, which is way below the estimates of MSFT.\\\\
To analyze the company's profitability more granuarly we look at other profitability measures (past 12 months in \% ) and compare them with the industry benchmark.
\begin{itemize}
\item Gross Margin: 67.8 (Benchmark: 30.6)
\item Net Margin: 31.0 (Benchmark: 6.9)
\end{itemize}

\noindent Note that MSFT outperforms the benchmark in every respective profitability measure, which usually constitutes a  buy-signal. \\\\
We take a look at the company's change in sales (\%), which declined during the past 12 months for the benchmark, since COVID-19 inhibited many consumers and business from usual consumption.
\begin{itemize}
	\item Sales: 14.0 (Benchmark: -1.7)
\end{itemize}
Note that MSFT very clearl outperformed the benchmark in sales.\\\\
\noindent If we want to make a serious buying-recommendation, we also have to look at the performance of the company within the \textbf{COVID-19} pandemic timeframe. Therefore, we analyze the earnings release of the second quarter of this year. The performance during Q2 might give a good indication of how the stock is going to perform for the rest of Q3 and Q4, since we will likely face similar consumer behaviour and will likely not have a vaccine or cure for COVID unitl 2021.\\\\
The earnings report for Q2 states: "Revenue increased \$4.4 billion or 14\%, driven by growth across each of our segments. Intelligent Cloud revenue increased, driven by server products and cloud services. Productivity and Business Processes revenue increased, driven by Office and LinkedIn. More Personal Computing revenue increased, driven by Windows, offset in part by a decrease in Gaming."\cite{microsoft}\\\\
"Gross margin increased \$4.5 billion or 22\%, driven by growth across each of our segments. Gross margin percentage increased, driven by sales mix shift to higher margin businesses. Gross margin included a 5-point improvement in commercial cloud, primarily from Azure."\cite{microsoft}\\\\
"Operating income increased \$3.6 billion or 35\%, driven by growth across each of our segments."\\ \cite{microsoft}\\\\
Overall the company was able to performed miraculously during these difficult times. Since COVID-19 forced many employees to work from home and pushed the digital transformation, services like cloud computing (MS Azure) get more and more important. As we see Microsoft was able to captalize on this. One can really see the high demand for cloud solutions if one looks at the growth of each segment seperately:\cite{microsoft}

\begin{table}[]
	\begin{tabular}{|l|l|l|}
		\hline
		\textit{\textbf{Segment}} & \textbf{revenue (\%)} & \textbf{operating income (\%)} \\ \hline
		Productivity and Business Processes & 17 & 29 \\ \hline
		Intelligent Cloud & 27 & 38 \\ \hline
		More Personal Computing & 2 & 41 \\ \hline
	\end{tabular}
\end{table}


\section{Financial Ratios}
Comparing financial ratio across stocks might give us a better idea how well the company is situated compared to benchmarks and other similar companies.

\begin{table}[h]
	\begin{tabular}{|l|l|l|l|l|}
		\hline
		\textit{\textbf{ratio/indicator}} & \textbf{MSFT} & \textbf{NASDAQ100} & \textbf{ORACLE} & \textbf{IBM} \\ \hline
		ROE (in \%) & 40.1 & 29 & 41.3 & 59.9 \\ \hline
		Quick Ratio & 2.49 & - & 3.02 & 0.93 \\ \hline
		Total Debt/Total Equity & 69.4 & - & 610.2 & 339.5 \\ \hline
		Working Capital Ratio & 2.516 & - & 3.02 & 0.968 \\ \hline
		P/E LTM & 37.2 & - & 18.1 & 13.9 \\ \hline
	\end{tabular}
\end{table}
\noindent Return on equity (ROE) measures how effectively management is using a company’s assets to create profits. One can observe that MSFT is at the lower end when comparing it with IBM or ORACLE. However, Microsoft still has a ROE, which is 10 percentage-points higher the average of NASDAQ100 index.The quick ratio of MSFT compares quite well to other competitors, which indicates that the company is capable to pay its current liabilities without needing to sell its inventory or get additional financing. This is a good sign for equity-investors. The debt-to-equity (D/E) ratio compares a company’s total liabilities to its shareholder equity and can be used to evaluate how much leverage a company is using. Note that Microsoft uses way less leverage than it competitors. This reduces the risk for equity investors. Next we look at the working capital of the company. Positive working capital indicates that a company can fund its current operations and invest in future activities and growth. One can observe that MSFT's working capital ratio is positive and sort of competitive when comparing it to similar comapnies. As the last ratio we like to analyze the company's P/E ratio (price-to-earnings ratio). The price-earnings ratio (P/E ratio) relates a company's share price to its earnings per share. A high P/E ratio could mean that a company's stock is over-valued, or else that investors are expecting high growth rates in the future. One can interpret this ratio in two ways, but if we consider the positive estimated growth of the analysists from above, we can conclude that it must be the latter of the two options.

\section{Final Recomendation}
With a market capitalisation of \$1.672(B) MSFT is a solid blue-chip stock, which is actively traded on multiple stock exchanges around the world, providing low liquidity-risk for seller/buyers. This is also seen when looking at the average daily volume of the last 3 months. The sharpe recovery from the pandemic-shock and the strong upward-trend, that does not seem to be bothered by the current situation, makes Microsoft a solid momentum stock, which investors should definately capitalize on. Experts and analysts see the stock at a target price of \$233.13, which is a clear sign that the stock is currently undervalued. But what about the earnings- and profitability factors of the company? - Earnings thrive during the COVID pandemic. The sales of MSFT are well above the average of the industry. Admittedly, the company's dividend yield is not as high as of some of its competitiors like IBM or ORACLE, however, its growth potential is high. The sales grew by approx. 14\% during the last 12 months and estimates of future EPS-figures are largely positive according to 31 analysts. These analysts predict a long-term growth rate of about 15.2\%. Moreover, the profitability figures of Microsoft  are well above its benchmark (see figures above) which adds substance to this buy-signal.\\\\
By capitalizing on the digital transformation, Microsoft is able to profit from a new way of business in the "stay-at-home"-economy by providing cloud solution like MS Azure, which make it possible to desentralize data-storage and computing operations of various businesses. With more and more people working from home, the demand for operating systems and new hardware (i.e. notebooks, laptops, etc.) is stable. The portfolio of Microsoft itself is well diversivied with other income-channels like LinkedIn and countless software solutions for all kinds of businesses (i.e. MS Office, VS Code, Visual Studio, GitHub, etc.). The company is a pandemic-proof stock and the second quarter of 2020 underpins this fact very clearly. With remote-work, parallel- and cloud-computing being the future of businesses, MSFT is here to stay and currently one of the best stocks you could have owned during the pandemic-crisis. The momentum of this stock is still pretty strong and from our point of view MSFT is still undervalued.\\\\
Taking all the facts mentioned above into account, Microsoft constitutes a clear "buy"-opportunity.



\bibliography{References}
\end{document}
